\documentclass[a4paper,twoside=false,abstract=false,numbers=noenddot,
titlepage=false,headings=small,parskip=half,version=last]{scrartcl}
\usepackage[utf8]{inputenc}
\usepackage[T1]{fontenc}
\usepackage[english]{babel}
\usepackage[colorlinks=true, pdfstartview=FitV,
linkcolor=black, citecolor=black, urlcolor=blue]{hyperref}
\usepackage{verbatim}
\usepackage{graphicx}
\usepackage{multirow}

\usepackage{tikz}
\usetikzlibrary{matrix}

\usepackage{amsmath}
\usepackage{amsthm}
\usepackage{amssymb}
\usepackage{amsfonts}

\usepackage{float}

\usepackage{gensymb}

\usepackage{authblk}

\usepackage{helpers}


\title{Suspensions with small, spherical particles}
\subtitle{SA104X Degree Project in Engineering Physics}
    \author[2]{Samuel Zackrisson$^1$ \\ \footnotesize Supervisor: Anna-Karin Tornberg}
    \affil[1,2]{Department for Numerical Analysis, Royal Institute of Technology}
    \affil[1]{samuelz@kth.se}
    \affil[2]{akto@kth.se}


\begin{document}
\maketitle
\thispagestyle{empty}

\begin{abstract}
Analytic solutions for systems of many particles in fluid dynamics and electrostatics are few and far-between.
Simulations are therefore essential to studying these systems.
This is commonly done without directly calculating the fluid's velocity field.
Calculating the field in a three dimensional setting on a cubic grid is $O(N^3)$, so finding efficient and accurate methods are of great interest.
In this report a method using the spectral accuracy of the fourier transform is studied.
The method is applied to a ball of particles sedimenting in a fluid affected by gravity.
The properties of this method are compared to similar studies of the same system.
The purpose is to map and evaluate the properties this method in comparison with present work.
\end{abstract}

\tableofcontents

\section{Introduction}
\subsection{Sedimenting suspensions of small particles}
Analytic solutions for systems of many particles in fluid dynamics and electrostatics are few and far-between. Simulations are therefore essential to studying these systems. This is commonly done without directly calculating the fluid's velocity field as in as in a three dimensional setting with N gridpoints along each dimension gives an $O(N^3)$ scale to the problem \cite{fallingclouds}. Finding fast methods of performing calculations on this large set can be simplified by making the perhaps controversial assumption of a periodic grid - assuming that the N$^3$-point cube is replicated much like the primitive cells of a crystal lattice. This introduces a periodicity which makes room for fourier methods and - for smooth functions - spectral (exponential) convergence in solutions and the use of the fast fourier transform, which in some places reduced the complexity to $O(NlogN)$ \cite{fluctuatesediment}. In this project a specific method of using the fourier transform is studied, applied to a ball of particles sedimenting in a fluid affected by gravity, and compared to similuar studies of the same system. The purpose is to evaluate this method in comparison with present work on the same simulation.

\subsection{Purpose of this study}
To implement a spectral method for simulation of small, spherical particles sedimenting in a fluid.\\
To verify some fundamental properties of the methods used.\\
To compare this method to other methods employed by others. This is done by looking at convergence, differences in flow etc.\\
To compare how this method introduces new and preserves macroscopic phenomena, compared to how the other methods behave.
\subsection{Delimitations}
We confine ourselves to microhydrodynamics. $Re<<1$.\\
I limit the point2grid interp to epsilon hats and delta blobs.\\
I limit the grid2point interp to some simple built-in function.\\
I limit the time step to Euler forward.\\
I limit the primitive periodic cell to a cubic NxNxN grid.\\
I do not go much into comparing the evolution of the point clouds.\\
\section{Theory}
\subsection{Stokes flow}
For the small particles in an incompressible, viscous fluid in this study, the Reynolds number is assumed to be small ($Re<<1$). The general Navier-Stokes equations can under these conditions be approximated as a linear set of equations, the steady Stokes equations. A solution to these equations is called as a \emph{Stokes flow}. For more details on the equations as they appear here, see \cite{guazzellibook}.\\
The continuity equation for an incompressible fluid appears as
\begin{equation}
\nabla \cdot \mathbf{u} = 0 \label{eq:incompressible}
\end{equation}
and the equations for conservation of momentum are
\begin{equation}
\mu \nabla^2 \mathbf{u} = \nabla p - \mathbf{f} \label{eq:stokesequations}
\end{equation}

Here $\mu$ is the viscosity of the fluid, $\mathbf{u}$ is the velocity field of the fluid, $p$ is the absolute pressure and $\mathbf{f}$ is the force on the fluid, the \emph{forcing function}. $\mathbf{u}$, $p$ and $\mathbf{f}$ are multivariate functions on $\mathbb{R}^3$.\\

In some methods a periodicity assumption is made to allow for fourier methods to be used efficiently,\footnote{This controversial assumption is expanded upon in the appendix.}
\begin{equation}
\mathbf{u}(\mathbf{r}) = \mathbf{u}(\mathbf{r}+L\mathbf{n}),\text{ }\forall \mathbf{n} \in \mathbb{Z}^3
\end{equation}
Thus if $\mathbf{u}$ is known in $[0,L)^3$, it is known in $\mathbb{R}^3$.\\
The set of equations in \eqref{eq:incompressible} and \eqref{eq:stokesequations} are solvable in $\mathbf{k}$-space for the fourier coefficients $\left\{\hat{\mathbf{u}}_\mathbf{k}\right\}$,
\begin{equation}
\hat{\mathbf{u}}_\mathbf{k} = \frac{1}{\nu k^2} \left( \hat{\mathbf{f}}_\mathbf{k} - \mathbf{k} \frac{\mathbf{k}\cdot \hat{\mathbf{f}}_\mathbf{k}}{k^2} \right) \label{eq:fourierstokessolution}
\end{equation}
where $\frac{1}{\mathbf{k}}$ is taken to be $0$ for $\mathbf{k}=(0,0,0)$. \eqref{eq:fourierstokessolution} is derived in the appendix.
\subsection{Discretization}
The velocity field $\mathbf{u}$ in $[0,L)^3$ is represented by a set of values $\mathbf{u}_\mathbf{m}:=\mathbf{u}(\mathbf{m}\cdot L/N)$, for $\mathbf{m}\in I = \left\{(i,j,k) | i,j,k=1,2,...,N\right\}$.
In other words the field $\mathbf{u}$ is described on a cubic $N\times N\times N$-grid.
The forcing function is analogously represented by the values $\mathbf{f}_\mathbf{m}, \mathbf{m}\in I$.\\
The Stokes equations are approximated by 

\subsection{Regularization}
The force of gravity on the particles gives a forcing function for Stokes equations,
\begin{align}
\mathbf{f}(\mathbf{x}) &= -g\hat{\mathbf{z}}\sum_{i=1}^M \delta (\mathbf{x}-\mathbf{x}_i)\label{eq:singularforcingfunction}\\
&= -g\hat{\mathbf{z}}\sum_{i=1}^M \delta^\epsilon (\mathbf{x}-\mathbf{x}_i)
\end{align}
To evaluate the singular function in \eqref{eq:singularforcingfunction} on the grid an interpolating function $\delta^\epsilon$ is introduced.
Some conditions are imposed on this funtion.
$\delta^\epsilon$ should sum to $1$ corresponding to $\int \delta(\mathbf{x}) d\mathbf{x}=1$.
$\delta^\epsilon$ should be as smooth as possible since this is the limiting factor on accuracy in \eqref{eq:fourierstokessolution}.
Two such choices of functions are discussed in this report, mainly based on the methods in \cite{spectralewald} and \cite{interfaceregularization}.
\subsubsection{Triangle functions}
\subsubsection{Cardinal splines}

\subsection{Other simulation methods}
To circumvent the problem of calculating the Stokes flow in the simulated region a method based on the Greens function can be used.
The solution to the Stokes equations in \eqref{eq:stokesequations} in $\mathbb{R}^3$ for the choice $\mathbf{f}(\mathbf{x}) = \delta(\mathbf{x})$ is the Stokeslet,
\begin{equation}
[Stokeslet]
\end{equation}
which gives a solution for $\mathbf{u}$ in \eqref{eq:singularforcingfunction},
\begin{equation}
[Solutionsum]
\end{equation}
...and maybe some hint at Ewald summation or similar.

\section{Setup and algorithms}
\subsection{The scene}
The system in question is a cloud of $M$ particles randomly placed within a sphere of radius $r_0$. This sphere begins in the center of a primitive cell, a cube of side length $L$, which is periodically repeated throughout space.\\
The particles are affected by a gravitational force downward and a lift from the fluid. The resulting force on the particles is the sole driving force in the simulation.\\
The force from the particles on the fluid make up the forcing function in Stokes' equation.
\subsection{The four-step simulation}

\image{solver-principle-transparent}{The simulation algorithm steps\label{fig:simprinciple}}
Each timestep the simulation has four distinct moments, as illustrated in figure \ref{fig:simprinciple}.

Step 1: Interpolating from particle positions to grid positions\\
The gravitational force on the particles manifests as singular functions in the Stokes' equation forcing function. To evaluate this force on the grid, the singular functions are regularized to the grid by representing them with regularized $\delta$-functions and evaluating these on the grid.\\ This interpolates the $M$ point forces to a forcing function on the $N^3$-grid.\\
Step 2: Solving Stokes' equation\\
Given the forcing function it is a matter of solving Stokes' equation. This is done using the fft (and thereby the periodicity assumption), giving the following formula derived in appendix.

Step 3: Interpolate the velocity field to the particle positions\\
This is simply done by cubic interpolation using Matlab's builtin function \emph{gridinterp}.\\
Step 4: Timestepping the particles\\
Given the particle velocities, the particles' positions can now be calculated using Euler's method.\\

If the only interest is studying the velocity field solution on the grid, steps 1-2 are sufficient. This can be done for any time $t$ in the simulation, as Stokes' equation is \emph{memoryless}, it does not depend on the history of neither the velocity field nor the forcing functions.

\subsection{The particle cloud setup used here}

\section{Results}

\subsection{Verifying some fundamental properties of the methods}
\subsubsection{Spectral convergence of solutions for smooth forcing functions}
\subsubsection{Convergence of solutions for regularized singular forcing functions}

At the moment the method with cardinal splines is not working, as can be seen in plot \ref{fig:failspline}. The upper data points show the convergence for the regularization using cardinal splines, which should be considerably faster than the second data set from the delta "hats". (Author's note: it's actually working now, but I have not produced any plots and comparisons yet.)
\plot{badsplines}{The upper segment is using splines, p=3 and 5 (overlapping). The lower segment is using epsilon hats, which clearly overtakes the splines when it comes to convergence, implying something is wrong with the implementation.\label{fig:failspline}}

\subsubsection{Convergence for regularization scaling with the grid}
\subsubsection{Moments of the regularized delta functions}
\subsubsection{Fourier Convergence results}
Fouriermetoder är effektiva vid låg upplösning tack vare spektral konvergens, eller exponentiell konvergens, för släta/glatta funktioner. Det går att visa [REF] att $c_k \approx |k|^{-n}$ för $f \in C^{n-1}$.
Med denna enklaste deltafunktions-utsmetning så bör fourier koefficienterna avta som $|k|^{-1}$.
Nedan ses vad som händer för fixerat epsilon. Jämför med senare 
konvergens hos hastighetsfältetslösningar
Låter vi epsilon ~ 1/N så avtar felnormen som ~ $N^{-2.97}$, medan epsilon fixt ger felet ~ $N^{-2.52}$. I det senare fallet fås tydligt inte den spektrala konvergens vi hade kunnat förvänta oss om de regulariserade deltafunktionerna hade varit glatta funktioner.
\plot{eps-conv/varying_deltas_2norms_small}{Normed successive differences in solutions as N increases and epsilon decreasing with it.}
\plot{eps-conv/fixed_deltas_2norms_small}{Normed successive differences in solutions as N increases, keeping epsilon fixed.}
\subsection{Comparing the solutions with other known methods}
\subsubsection{Scaling the primitive periodic cell}\label{cellscale}
Unlike the system studied by Metzger et al \cite{fallingclouds}, the system studied in this article is periodic. If the cloud of particles is too large compared to the period of the system it is not unlikely to have interactions between clouds of different primitive cells. This can be studied by scaling the length of the primitive cells and looking at the differences between successive solutions for the velocity field in a cube of fixed size containing the particle cloud. The error decreases as ~ $L^{-0.87}$.
\plot{len-conv/lengthvary_small}{Normed successive differences in solutions as the primitive cell increases.}

\subsection{Macroscopic phenomena}
The evolution of the cloud shape in a periodic box appears to agree with that observed by Metzger et al \cite{fallingclouds} in a non-periodic setting. The sedimenting ball becomes oblate, forms a torus and eventually breaks into two smaller clouds. This can be confirmed in more detail by studying e.g. the particle density along the z-axis in the cloud center, the pressure gradient in and surrounding the cloud, the velocity field in and around the cloud.\\
For too few gridpoints the cloud appears to take a square shape, aligned with the grid cells. When this effect in practice disappears can be studied further, e.g. by plotting the particle density as a function of azimuthal angle.\\

\section{Discussion}

\section{Conclusions}


\bibliography{bibliography}
\bibliographystyle{plain}

\section{Appendices}
\subsection{Fourier series}
The Fourier series coefficient $\mathcal{F}(f)_{\mathbf{k}}$ of a function $f$ on the volume $V=[0,L)^3$ is taken to be
\begin{align}
\mathcal{F}(f)_{\mathbf{k}} &= \frac{1}{L^3} \int_V e^{-\frac{2\pi}{L}i(\mathbf{k}\cdot \mathbf{x})}f(\mathbf{x})d\mathbf{x}
    \label{eq:fourierdef}
\end{align}
The coefficients for the gradient coefficients are
\begin{align}
\mathcal{F}(\nabla f)_{\mathbf{k}} &= \frac{2\pi}{L}i\mathbf{k} \mathcal{F}(f)_{\mathbf{k}}
\end{align}
and so Stoke's equation can be written in fourier space as \\
$\left(STUFF\right)$\\
...here \eqref{eq:fourierdef} will be discretized to show how it relates to the DFT in Matlab \cite{matlab}.

\subsubsection{Solving differential equations using the fft}
(Also scaling the fourier transform, for use in \ref{cellscaling})
\subsection{Stuff I did not want cluttering the main text}
Periodicity assumption\\
Moments and delta function regularization
\end{document}
