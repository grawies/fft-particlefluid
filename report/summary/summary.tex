\documentclass[a4paper,
%twocolumn,
twoside=false,abstract=false,numbers=noenddot,
titlepage=false,headings=small,parskip=half,version=last]{scrartcl}
\usepackage[english]{babel}
\usepackage[utf8]{inputenc}
\begin{document}
{\Large \bf Sammanfattning}\\
Det finns sällan vare sig tillräckligt med beräkningskraft eller analytiska lösningar för stora flerpartikelsystem i fluidsmekanik och elektrostatik.
Simuleringar och numeriska approximationer är därför grundläggande metoder för att studera dessa system.
Partiklarnas banor beräknas vanligen utan att direkt beräkna fluidens hastighetsfält.
I denna rapport studeras en simuleringsmetod som utnyttjar den spektrala noggrannheten hos fouriertransformen för att finna partikelhastigheterna via fluidens hastighetsfält.
Metoden tillämpas på en periodisk kub med en suspension av små, sfäriska partiklar påverkade av gravitationen i en fluid i ett försök att efterlikna beteendet hos ett likadant, icke-periodiska system.
Resultat för fåpartikelsystem förklarar kvalitativt formen på lösningsströmningar i förhållande till valet av inerpolation mellan partikelpositioner och rutnät, samt kvantitativt kartlägger vissa konvergensegenskaper hos en viss klass av interpolerande funktioner, cardinal B-splines.
Egenskaperna hos denna metod på det periodiska systemet studeras och jämförs med en liknande studie av det icke-periodiska systemet för många, $\sim 1000$, partiklar.

\vspace{20mm}
{\Large \bf Summary}\\
Feasibly computable analytic solutions for systems of many particles in fluid dynamics and electrostatics are few and far-between.
Simulations and numerical approximations are essential to studying these systems.
This is commonly done without directly calculating the interacting field between particles.
In this report a method utilizing the spectral accuracy of the Fourier transform is studied to calculate particle velocities via the surrounding fluid velocity field.
The method is applied to a periodic cube of a suspension of small, spherical particles sedimenting in a fluid affected by gravity, in an attempt to mimic the behaviour of a similar infinite system.
Results for a few particles qualitatively relate the shape of the solution to the choice of interpolation between particles to grid and quantitatively maps some convergence properties of a certain class of interpolating functions, cardinal B-splines.
The properties of the method on the periodic system are also examined and compared to a similar study of the infinite system for many, $\sim 1000$, particles.
\end{document}
