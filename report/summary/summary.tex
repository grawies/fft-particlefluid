\documentclass[a4paper,
%twocolumn,
twoside=false,abstract=false,numbers=noenddot,
titlepage=false,headings=small,parskip=half,version=last]{scrartcl}
\usepackage[english]{babel}
\usepackage[utf8]{inputenc}
\begin{document}
{\Large \bf Sammanfattning}\\
Det finns sällan vare sig tillräckligt med beräkningskraft eller analytiska lösningar för stora flerpartikelsystem i fluidsmekanik och elektrostatik.
Simuleringar och numeriska approximationer är därför grundläggande metoder för att studera dessa system.
Partiklarnas banor beräknas vanligen utan att direkt beräkna fluidens hastighetsfält.
Att enbart lagra ett diskretiserat fält i tre dimensioner på ett kubiskt $N\times N\times N$-rutnät har rumskomplexiteten $O(N^3)$.
Det är därför av stort intresse att finna effektiva och noggranna algoritmer.
I denna rapport studeras en simuleringsmetod som utnyttjar den spektrala noggrannheten hos fouriertransformen.
Metoden tillämpas på en periodisk kub med en suspension av små, sfäriska partiklar påverkade av gravitationen i en fluid i ett försök att efterlikna beteendet hos samma, icke-periodiska system.
Egenskaperna hos denna metod på det periodiska systemet studeras och jämförs med liknande studier av det icke-periodiska systemet.

\vspace{20mm}
{\Large \bf Summary}\\
Feasibly computable analytic solutions for systems of many particles in fluid dynamics and electrostatics are few and far-between.
Simulations and numerical approximations are essential to studying these systems.
This is commonly done without directly calculating the interacting field between particles.
Simply storing a discretized field in a three dimensional setting on a cubic $N\times N\times N$-grid is $O(N^3)$, so finding efficient and accurate methods are of great interest.
In this report a method utilizing the spectral accuracy of the Fourier transform is studied.
The method is applied to a periodic cube of a suspension of small, spherical particles sedimenting in a fluid affected by gravity, in an attempt to mimic the behaviour of the infinite system.
The properties of this method on the periodic system are examined and compared to similar studies of the infinite system.

\end{document}
